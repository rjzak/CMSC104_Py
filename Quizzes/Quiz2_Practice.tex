% Copyright 2024 Richard J. Zak
% richard.j.zak@gmail.com

\documentclass[letter,10pt]{article}
\usepackage[breaklinks]{hyperref}
\hypersetup{
    bookmarks=true,         % show bookmarks bar?
    unicode=false,          % non-Latin characters in Acrobat’s bookmarks
    pdftoolbar=true,        % show Acrobat’s toolbar?
    pdfmenubar=true,        % show Acrobat’s menu?
    pdffitwindow=false,     % window fit to page when opened
    pdfstartview={XYZ null null 1.00},    % disable zoom
    pdftitle={Practice Quiz 2},    % title
    pdfauthor={Richard Zak},     % author
    pdfsubject={UMBC CMSC104 Problem Solving and Computer Programming},   % subject of the document
    pdfkeywords={Computer Science, Programming, Problem Solving, CSEE}, % list of keywords
    pdfnewwindow=true,      % links in new PDF window
    colorlinks=false,       % false: boxed links; true: colored links
    linkcolor=red,          % color of internal links (change box color with linkbordercolor)
    citecolor=green,        % color of links to bibliography
    filecolor=magenta,      % color of file links
    urlcolor=cyan           % color of external links
}
\usepackage{graphicx}
\usepackage{fancyhdr}
\usepackage{multicol}
\pagestyle{fancy}
\usepackage[letterpaper, margin=1in]{geometry}
\geometry{letterpaper}
\usepackage{listings} % Syntax highlighing
\usepackage{amsmath}
\usepackage{parskip} % Disable initial indent
\usepackage{color,soul} % Highligher
\usepackage[normalem]{ulem} % Strikethrough with \sout{}

\definecolor{codegreen}{rgb}{0,0.6,0}
\definecolor{codegray}{rgb}{0.5,0.5,0.5}
\definecolor{codepurple}{rgb}{0.58,0,0.82}
\definecolor{backcolour}{rgb}{0.97,0.97,0.97}

\lstdefinestyle{mystyle}{
    backgroundcolor=\color{backcolour},
    commentstyle=\color{codegreen},
    keywordstyle=\color{magenta},
    numberstyle=\tiny\color{codegray},
    stringstyle=\color{codepurple},
    basicstyle=\ttfamily\small,
    breakatwhitespace=false,
    breaklines=true,
    captionpos=b,
    keepspaces=true,
    numbers=left,
    numbersep=5pt,
    showspaces=false,
    showstringspaces=false,
    showtabs=false,
    tabsize=2
}

\lstset{style=mystyle}

\usepackage[utf8]{inputenc}
\fancyhf{}
\renewcommand{\headrulewidth}{0pt} % Remove default underline from header package
\rhead{CMSC 104 Section 01: Practice Quiz 2}
%\rhead{}
\lhead{\begin{picture}(0,0) \put(0,-10){\includegraphics[width=1.1cm]{../Images/UMBC-vertical}} \end{picture}}
\cfoot{\thepage}
\rfoot{Spring 2025
}
\lfoot{CMSC 104 Section 01}
\AtEndDocument{\vfill \footnotesize{Last modified: 12 October 2024}}
\AtEndDocument{\rfoot{Spring 2025
}}
\renewcommand\thesubsection{\arabic{subsection}} % Show only subsection numbers, not section.subsection

\begin{document}

\huge
\textbf{Practice Quiz 2}
\normalsize

\paragraph{}Quiz 2 covers chapters 1 through 6. Please answer the following questions. Partial credit may be given for incomplete free response questions.

\begin{enumerate}
    \item Accessing a single character in a string is called what?
    \begin{enumerate}
        \item Slicing
        \item Indexing
        \item Concatenating
        \item Assigning
    \end{enumerate}

    \item Merging two or more strings into a larger string is called what?
    \begin{enumerate}
        \item Slicing
        \item Indexing
        \item Concatenating
        \item Assigning
    \end{enumerate}

    \item Getting a sub-string from a larger string ("science" from "Computer science is fun") is called what?
    \begin{enumerate}
        \item Slicing
        \item Indexing
        \item Concatenating
        \item Assigning
    \end{enumerate}

    \item True or False: A function in Python always returns a value.

    \item Identify four things wrong with this function:
    \begin{lstlisting}[language=python]
# The Leibniz forumula for Pi
def calculate_pi(iterations=1000000):
    k = 0 # Denominator

    for i in range(iterations):
        if i % 2 = 0:
            s += 4/k
        else:
            s -= 4/k
        k += 2
    pi = s

# Print the results from the function
print("Pi with 10 iterations:", calculate_pi(10))
print("Pi with 1000000 iterations:", calculate_pi())
    \end{lstlisting}

    \item Which are benefits of writing functions?
    \begin{enumerate}
        \item Enables code reuse.
        \item Helps improve the organization of code within a program.
        \item Makes the code more readable, and therefore easier to understand.
        \item Is required to work on Windows systems.
        \item May reduce the mount of code in a program.
        \item May make the program easier to maintain.
        \item Allows the program run faster.
        \item Functions negate the need for commenting of code.
    \end{enumerate}

    \item Under what condition(s) may a Python function change the value of the parameter it receives?

    \item How many values may a Python function return?

    \item Implement a tip calculator using at least one function.

    \item Write a code snippet which opens a file called \texttt{foo.txt} and prints each line.

\end{enumerate}

\end{document}