% Copyright 2024 Richard J. Zak
% richard.j.zak@gmail.com

\documentclass[letter,10pt]{article}
\usepackage[breaklinks]{hyperref}
\hypersetup{
    bookmarks=true,         % show bookmarks bar?
    unicode=false,          % non-Latin characters in Acrobat’s bookmarks
    pdftoolbar=true,        % show Acrobat’s toolbar?
    pdfmenubar=true,        % show Acrobat’s menu?
    pdffitwindow=false,     % window fit to page when opened
    pdfstartview={XYZ null null 1.00},    % disable zoom
    pdftitle={Practice Quiz 1 Answers},    % title
    pdfauthor={Richard Zak},     % author
    pdfsubject={UMBC CMSC104 Problem Solving and Computer Programming},   % subject of the document
    pdfkeywords={Computer Science, Programming, Problem Solving, CSEE}, % list of keywords
    pdfnewwindow=true,      % links in new PDF window
    colorlinks=false,       % false: boxed links; true: colored links
    linkcolor=red,          % color of internal links (change box color with linkbordercolor)
    citecolor=green,        % color of links to bibliography
    filecolor=magenta,      % color of file links
    urlcolor=cyan           % color of external links
}
\usepackage{graphicx}
\usepackage{fancyhdr}
\usepackage{multicol}
\pagestyle{fancy}
\usepackage[letterpaper, margin=1in]{geometry}
\geometry{letterpaper}
\usepackage{listings} % Syntax highlighing
\usepackage{amsmath}
\usepackage{parskip} % Disable initial indent
\usepackage{color,soul} % Highligher
\usepackage[normalem]{ulem} % Strikethrough with \sout{}
\usepackage{bbding} % For checkmark in itemize
\newcommand*\tick{\item[\Checkmark]}
\newcommand*\fail{\item[\XSolidBrush]}

\definecolor{codegreen}{rgb}{0,0.6,0}
\definecolor{codegray}{rgb}{0.5,0.5,0.5}
\definecolor{codepurple}{rgb}{0.58,0,0.82}
\definecolor{backcolour}{rgb}{0.97,0.97,0.97}

\lstdefinestyle{mystyle}{
    backgroundcolor=\color{backcolour},
    commentstyle=\color{codegreen},
    keywordstyle=\color{magenta},
    numberstyle=\tiny\color{codegray},
    stringstyle=\color{codepurple},
    basicstyle=\ttfamily\small,
    breakatwhitespace=false,
    breaklines=true,
    captionpos=b,
    keepspaces=true,
    numbers=left,
    numbersep=5pt,
    showspaces=false,
    showstringspaces=false,
    showtabs=false,
    tabsize=2
}

\lstset{style=mystyle}

\usepackage[utf8]{inputenc}
\fancyhf{}
\renewcommand{\headrulewidth}{0pt} % Remove default underline from header package
\rhead{CMSC 104 Section 02
: Practice Quiz 1 Answers}
%\rhead{}
\lhead{\begin{picture}(0,0) \put(0,-10){\includegraphics[width=1.1cm]{../Images/UMBC-vertical}} \end{picture}}
\cfoot{\thepage}
\rfoot{Spring 2025
}
\lfoot{CMSC 104 Section 02
}
\AtEndDocument{\vfill \footnotesize{Last modified: 16 September 2024}}
\AtEndDocument{\rfoot{Spring 2025
}}
\renewcommand\thesubsection{\arabic{subsection}} % Show only subsection numbers, not section.subsection

\begin{document}

\huge
\textbf{Practice Quiz 1 with Answers}
\normalsize

\paragraph{}Please answer the following questions. Partial credit may be given for incomplete free response questions.

\begin{enumerate}
    \item Which programming language are we going to learn in this course?
    \begin{enumerate}
        \tick Python
        %\item C
        %\item C++
        %\item Java
    \end{enumerate}
    
    \item A keyboard is considered a type of \underline{~~ input ~~} device, while a hard drive is considered a type of \underline{~~ storage ~~} device.
    
    \item Why does adding more RAM not always make computers faster? \\
    \textit{RAM can make the computer faster for some programs, but for really intense programs, like modern video games, there's no substitute for a faster processor or better graphics card. Ultimately, it is the CPU which sets the speed of the computer.}
    
    \item Provide three examples of I/O devices.
    \begin{enumerate}
        \item \textit{Mouse, input device}
        \item \textit{Microphone, input device}
        \item \textit{Speakers, output device}
    \end{enumerate}
    
    \item Which data type would you use to represent currency, and why? \\
    \textit{Use a float so that subunits, like cents, may be represented.}
    
    \item What is the difference between syntax and semantics? \\
    \textit{Syntax is the structure of the language, semantics is what's being conveyed by the language.}
    
    \item Write a Python script which converts Fahrenheit to Celsius. $\dfrac{F - 32}{\dfrac{9}{5}}$
    \begin{lstlisting}[language=python]
celsius = (fahrenheit - 32) / (9 / 5)
    \end{lstlisting}
    
    \item Explain why the following joke is funny: \\
    ``There are 10 kinds of people in this world, those who understand binary and those who don't.'' \\
    \textit{This is funny because 10 looks like the decimal number ten, but it also means two in binary. Also, binary has only two states, just like the state of understanding binary or not.}
    
    \item Which of the following variable names are \underline{not} permitted?
    \begin{itemize}
        \fail auc \textit{This is a valid variable name.}
        \tick 4real \textit{This is invalid because a variable name may not start with a number.}
        \fail se7en \textit{This is valid, a variable name may have an integer in it.}
        \fail \_cool \textit{This is valid, an underscore may be in a variable name.}
        \tick -cats \textit{This is invalid, as Python will think you want the negative version of the variable `cats`.}.
        \tick a bc \textit{This is invalid, as a variable may not have a space.}
    \end{itemize}

    \item What is the output from this script?
    \begin{lstlisting}[language=python]
for i in range(0):
    print("Hello!")
    \end{lstlisting}
    \textit{Nothing, this has no output as this loop runs zero times, so the print statement is never executed.}
    
    \item What is an algorithm? \\
    \textit{An algorithm is a finite set of step-by-step repeatable \& unambiguous instructions for solving a problem.}
    
    \item Write pseudocode for a tip calculator, where the user would provide the sub-total and tip percentage, and the program displays the final total.
    \begin{verbatim}
DISPLAY "Enter subtotal"
READ <subtotal>

DISPLAY "Enter tip percentage"
READ <tip>

final_total = subtotal * tip + subtotal
DISPLAY "The bill is: ", final_total
    \end{verbatim}
    
    \textit{Recall that pseudocode is a way to express what an algorithm does without requiring the reader to know a specific programming language to understand it. It also helps with planning, so you can think about the algorithm itself before having to worry about programming language syntax.}
    
    \item There are a few differences between compiled and interpreted languages.
    \begin{enumerate}
    	\item Which type of language is Python? \\
			\textit{Interpreted}
		\item What are two (or more) differences between these types of languages? \\
			\textit{Interpreted languages are slower and don't have to be compiled, so development may be faster. Interpreted languages need the interpreter installed, so you have to have Python installed to run Python code.} \\
			\textit{Compiled languages typically run faster, but have to be compiled for any code changes. Compiled languages don't need to have the compiler installed to be run.}
    \end{enumerate}
\end{enumerate}


\end{document}