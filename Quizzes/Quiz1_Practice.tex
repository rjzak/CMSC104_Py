% Copyright 2024 Richard J. Zak
% richard.j.zak@gmail.com

\documentclass[letter,10pt]{article}
\usepackage[breaklinks]{hyperref}
\hypersetup{
    bookmarks=true,         % show bookmarks bar?
    unicode=false,          % non-Latin characters in Acrobat’s bookmarks
    pdftoolbar=true,        % show Acrobat’s toolbar?
    pdfmenubar=true,        % show Acrobat’s menu?
    pdffitwindow=false,     % window fit to page when opened
    pdfstartview={XYZ null null 1.00},    % disable zoom
    pdftitle={Practice Quiz 1},    % title
    pdfauthor={Richard Zak},     % author
    pdfsubject={UMBC CMSC104 Problem Solving and Computer Programming},   % subject of the document
    pdfkeywords={Computer Science, Programming, Problem Solving, CSEE}, % list of keywords
    pdfnewwindow=true,      % links in new PDF window
    colorlinks=false,       % false: boxed links; true: colored links
    linkcolor=red,          % color of internal links (change box color with linkbordercolor)
    citecolor=green,        % color of links to bibliography
    filecolor=magenta,      % color of file links
    urlcolor=cyan           % color of external links
}
\usepackage{graphicx}
\usepackage{fancyhdr}
\usepackage{multicol}
\pagestyle{fancy}
\usepackage[letterpaper, margin=1in]{geometry}
\geometry{letterpaper}
\usepackage{listings} % Syntax highlighing
\usepackage{amsmath}
\usepackage{parskip} % Disable initial indent
\usepackage{color,soul} % Highligher
\usepackage[normalem]{ulem} % Strikethrough with \sout{}

\definecolor{codegreen}{rgb}{0,0.6,0}
\definecolor{codegray}{rgb}{0.5,0.5,0.5}
\definecolor{codepurple}{rgb}{0.58,0,0.82}
\definecolor{backcolour}{rgb}{0.97,0.97,0.97}

\lstdefinestyle{mystyle}{
    backgroundcolor=\color{backcolour},
    commentstyle=\color{codegreen},
    keywordstyle=\color{magenta},
    numberstyle=\tiny\color{codegray},
    stringstyle=\color{codepurple},
    basicstyle=\ttfamily\small,
    breakatwhitespace=false,
    breaklines=true,
    captionpos=b,
    keepspaces=true,
    numbers=left,
    numbersep=5pt,
    showspaces=false,
    showstringspaces=false,
    showtabs=false,
    tabsize=2
}

\lstset{style=mystyle}

\usepackage[utf8]{inputenc}
\fancyhf{}
\renewcommand{\headrulewidth}{0pt} % Remove default underline from header package
\rhead{CMSC 104 Section 02
: Practice Quiz 1}
%\rhead{}
\lhead{\begin{picture}(0,0) \put(0,-10){\includegraphics[width=1.1cm]{../Images/UMBC-vertical}} \end{picture}}
\cfoot{\thepage}
\rfoot{Spring 2025
}
\lfoot{CMSC 104 Section 02
}
\AtEndDocument{\vfill \footnotesize{Last modified: 20 August 2024}}
\AtEndDocument{\rfoot{Spring 2025
}}
\renewcommand\thesubsection{\arabic{subsection}} % Show only subsection numbers, not section.subsection

\begin{document}

\huge
\textbf{Practice Quiz 1}
\normalsize

\paragraph{}Please answer the following questions. Partial credit may be given for incomplete free response questions.

\begin{enumerate}
    \item Which programming language are we going to learn in this course?
    \begin{enumerate}
        \item Python
        \item C
        \item C++
        \item Java
    \end{enumerate}
    
    \item A keyboard is considered a type of \underline{~~ ~~ ~~ ~~ ~~ ~~ ~~ ~~ ~~ ~~} device, while a hard drive is considered a type of \underline{~~ ~~ ~~ ~~ ~~ ~~ ~~ ~~ ~~ ~~} device.
    
    \item Why does adding more RAM not always make computers faster?
    
    \item Provide three examples of I/O devices.
    \begin{enumerate}
        \item ~~ ~~
        \item ~~ ~~
        \item ~~ ~~
    \end{enumerate}
    
    \item Which data type would you use to represent currency, and why?
    
    \item What is the difference between syntax and semantics?
    
    \item Write a Python script which converts Fahrenheit to Celsius. $\dfrac{F - 32}{\dfrac{9}{5}}$
    
    \item Explain why the following joke is funny: \\
    ``There are 10 kinds of people in this world, those who understand binary and those who don't.''
    
    \item Which of the following variable names are \underline{not} permitted?
    \begin{itemize}
        \item auc
        \item 4real
        \item se7en
        \item \_cool
        \item -cats
        \item a bc
    \end{itemize}

    \item What is the output from this script?
    \begin{lstlisting}[language=python]
for i in range(0):
    print("Hello!")
    \end{lstlisting}
    
    \item What is an algorithm?
    
    \item Write pseudocode for a tip calculator, where the user would provide the sub-total and tip percentage, and the program displays the final total.
    
    \item There are a few differences between compiled and interpreted languages.
    \begin{enumerate}
    	\item Which type of language is Python?
		\item What are two (or more) differences between these types of languages?
    \end{enumerate}
\end{enumerate}


\end{document}