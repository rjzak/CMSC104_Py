% Copyright 2024 Richard J. Zak
% richard.j.zak@gmail.com

\documentclass[letter,10pt]{article}
\usepackage[breaklinks]{hyperref}
\hypersetup{
    bookmarks=true,         % show bookmarks bar?
    unicode=false,          % non-Latin characters in Acrobat’s bookmarks
    pdftoolbar=true,        % show Acrobat’s toolbar?
    pdfmenubar=true,        % show Acrobat’s menu?
    pdffitwindow=false,     % window fit to page when opened
    pdfstartview={XYZ null null 1.00},    % disable zoom
    pdftitle={Practice Quiz 2},    % title
    pdfauthor={Richard Zak},     % author
    pdfsubject={UMBC CMSC104 Problem Solving and Computer Programming},   % subject of the document
    pdfkeywords={Computer Science, Programming, Problem Solving, CSEE}, % list of keywords
    pdfnewwindow=true,      % links in new PDF window
    colorlinks=false,       % false: boxed links; true: colored links
    linkcolor=red,          % color of internal links (change box color with linkbordercolor)
    citecolor=green,        % color of links to bibliography
    filecolor=magenta,      % color of file links
    urlcolor=cyan           % color of external links
}
\usepackage{graphicx}
\usepackage{fancyhdr}
\usepackage{multicol}
\pagestyle{fancy}
\usepackage[letterpaper, margin=1in]{geometry}
\geometry{letterpaper}
\usepackage{listings} % Syntax highlighing
\usepackage{amsmath}
\usepackage{parskip} % Disable initial indent
\usepackage{color,soul} % Highligher
\usepackage[normalem]{ulem} % Strikethrough with \sout{}
\usepackage{bbding} % For checkmark in itemize
\newcommand*\tick{\item[\Checkmark]}
\newcommand*\fail{\item[\XSolidBrush]}

\definecolor{codegreen}{rgb}{0,0.6,0}
\definecolor{codegray}{rgb}{0.5,0.5,0.5}
\definecolor{codepurple}{rgb}{0.58,0,0.82}
\definecolor{backcolour}{rgb}{0.97,0.97,0.97}

\lstdefinestyle{mystyle}{
    backgroundcolor=\color{backcolour},
    commentstyle=\color{codegreen},
    keywordstyle=\color{magenta},
    numberstyle=\tiny\color{codegray},
    stringstyle=\color{codepurple},
    basicstyle=\ttfamily\small,
    breakatwhitespace=false,
    breaklines=true,
    captionpos=b,
    keepspaces=true,
    numbers=left,
    numbersep=5pt,
    showspaces=false,
    showstringspaces=false,
    showtabs=false,
    tabsize=2
}

\lstset{style=mystyle}

\usepackage[utf8]{inputenc}
\fancyhf{}
\renewcommand{\headrulewidth}{0pt} % Remove default underline from header package
\rhead{CMSC 104 Section 02
: Practice Quiz 2}
%\rhead{}
\lhead{\begin{picture}(0,0) \put(0,-10){\includegraphics[width=1.1cm]{../Images/UMBC-vertical}} \end{picture}}
\cfoot{\thepage}
\rfoot{Spring 2025
}
\lfoot{CMSC 104 Section 02
}
\AtEndDocument{\vfill \footnotesize{Last modified: 12 October 2024}}
\AtEndDocument{\rfoot{Spring 2025
}}
\renewcommand\thesubsection{\arabic{subsection}} % Show only subsection numbers, not section.subsection

\begin{document}

\huge
\textbf{Practice Quiz 2}
\normalsize

\paragraph{}Quiz 2 covers chapters 1 through 6. Please answer the following questions. Partial credit may be given for incomplete free response questions.

\begin{enumerate}
    \item Accessing a single character in a string is called what?
    \begin{enumerate}
        \tick Indexing
    \end{enumerate}

    \item Merging two or more strings into a larger string is called what?
    \begin{enumerate}
        \tick Concatenating
    \end{enumerate}

    \item Getting a sub-string from a larger string ("science" from "Computer science is fun") is called what?
    \begin{enumerate}
        \tick Slicing
    \end{enumerate}

    \item \textbf{True}: A function in Python always returns a value. \\
    \textit{A function returns one or more values using the \texttt{return} keyword, or else the function returns \texttt{None}.}

    \item Identify four things wrong with this function:
    \begin{lstlisting}[language=python]
# The Leibniz forumula for Pi
def calculate_pi(iterations=1000000):
    k = 0 # Denominator
    # initialization of `s` is missing.
    for i in range(iterations):
        if i % 2 = 0: # Should be the equality operator `==` not the assignment operator `=`
            s += 4/k # Division by zero, since `k` is initialized to zero
        else:
            s -= 4/k
        k += 2
    pi = s # The print statement expects this function to return a value, but there's no `return pi` statement.

# Print the results from the function
print("Pi with 10 iterations:", calculate_pi(10))
print("Pi with 1000000 iterations:", calculate_pi())
    \end{lstlisting}

    \item Which are benefits of writing functions?
    \begin{enumerate}
        \tick Enables code reuse. \\
        \textit{Think about the \texttt{print()} or \texttt{math.sqrt()} functions, we can use them despite not having written them ourselves, or even knowing how they work.}
        \tick Helps improve the organization of code within a program.
        \tick Makes the code more readable, and therefore easier to understand.
        \fail Is required to work on Windows systems.
        \tick May reduce the mount of code in a program.
        \tick May make the program easier to maintain. \\
        \textit{Better code organization should make it easier to understand.}
        \fail Allows the program run faster.
        \fail Functions negate the need for commenting of code. \\
        \textit{Comments are important, especially for complicated algorithms}.
    \end{enumerate}

    \item Under what condition(s) may a Python function change the value of the parameter it receives? \\
    \textit{Passing elements to a function in a list allow the function to update values such that the calling function is able to observe the new values. The same thing goes for \texttt{class} types, which we'll cover in Chapter 10.}

    \item How many values may a Python function return? \\
    \textit{At least one, the \texttt{None} type if there's no \texttt{return} statement. Otherwise, a function may return one or more values separated by a comma.}

    \item Implement a tip calculator using at least one function. \\
    \textit{This is a bit over-engineered compared to that's required, but this is a nice example since it uses a loop to ensure that the user enters a proper value, and the only way to exit the loop is a correct value. It makes good use of a function since it receives the message, so it works for both getting the subtotal and tip amount. This also uses an if statement to handle a tip amount expressed as a decimal or percentage, 0.1 vs. 10\%.}
    \begin{lstlisting}[language=python]
def getUserInput(input_message):
    while True:
        received = input(input_message)
        try:
            number = float(received)
            return number
        except:
            print("Sorry, {} is not a valid number. Please try again.".format(received))

subtotal = getUserInput("Enter the subtotal")
tip = getUserInput("Enter the tip")
if tip > 1.0:
    tip = subtotal / tip
else:
    tip = subtotal * tip
final_total = subtotal + tip
print("The total bill: ${:.2f}.".format(final_total))
    \end{lstlisting}

    \item Write a code snippet which opens a file called \texttt{foo.txt} and prints each line.
    \begin{lstlisting}[language=python]
f = open("foo.txt")
for line in f:
    print(line)
    \end{lstlisting}

\end{enumerate}

\end{document}