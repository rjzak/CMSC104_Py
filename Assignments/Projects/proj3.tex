% Copyright 2020-2024 Richard J. Zak
% richard.j.zak@gmail.com

\documentclass[letter,10pt]{article}
\usepackage[breaklinks]{hyperref}
\hypersetup{
    bookmarks=true,         % show bookmarks bar?
    unicode=false,          % non-Latin characters in Acrobat’s bookmarks
    pdftoolbar=true,        % show Acrobat’s toolbar?
    pdfmenubar=true,        % show Acrobat’s menu?
    pdffitwindow=false,     % window fit to page when opened
    pdfstartview={XYZ null null 1.00},    % disable zoom
    pdftitle={Project 3},    % title
    pdfauthor={Richard Zak},     % author
    pdfsubject={UMBC CMSC104 Problem Solving and Computer Programming},   % subject of the document
    pdfkeywords={Computer Science, Programming, Problem Solving, CSEE}, % list of keywords
    pdfnewwindow=true,      % links in new PDF window
    colorlinks=false,       % false: boxed links; true: colored links
    linkcolor=red,          % color of internal links (change box color with linkbordercolor)
    citecolor=green,        % color of links to bibliography
    filecolor=magenta,      % color of file links
    urlcolor=cyan           % color of external links
}
\usepackage{graphicx}
\usepackage{fancyhdr}
\usepackage{multicol}
\pagestyle{fancy}
\usepackage[letterpaper, margin=1in]{geometry}
\geometry{letterpaper}
\usepackage{attachfile}
\usepackage{listings} % Syntax highlighing
\usepackage{xcolor}
\usepackage{parskip} % Disable initial indent
\usepackage{color,soul} % Highligher
\usepackage[normalem]{ulem} % Strikethrough with \sout{}

\definecolor{codegreen}{rgb}{0,0.6,0}
\definecolor{codegray}{rgb}{0.5,0.5,0.5}
\definecolor{codepurple}{rgb}{0.58,0,0.82}
\definecolor{backcolour}{rgb}{0.97,0.97,0.97}

\lstdefinestyle{mystyle}{
    backgroundcolor=\color{backcolour},
    commentstyle=\color{codegreen},
    keywordstyle=\color{magenta},
    numberstyle=\tiny\color{codegray},
    stringstyle=\color{codepurple},
    basicstyle=\ttfamily\small,
    breakatwhitespace=false,
    breaklines=true,
    captionpos=b,
    keepspaces=true,
    numbers=left,
    numbersep=5pt,
    showspaces=false,
    showstringspaces=false,
    showtabs=false,
    tabsize=2
}

\lstset{style=mystyle}

\usepackage[utf8]{inputenc}
\fancyhf{}
\renewcommand{\headrulewidth}{0pt} % Remove default underline from header package
\rhead{CMSC 104 Section 01: Project 3}
%\rhead{}
\lhead{\begin{picture}(0,0) \put(0,-10){\includegraphics[width=1.1cm]{../../Images/UMBC-vertical}} \end{picture}}
\cfoot{\thepage}
\rfoot{Spring 2025
}
\lfoot{CMSC 104 Section 01}
\AtEndDocument{\vfill \footnotesize{Last modified: 03 August 2024}}
\AtEndDocument{\rfoot{Spring 2025
}}
\renewcommand\thesubsection{\arabic{subsection}} % Show only subsection numbers, not section.subsection

\begin{document}
    
    \huge
    \textbf{Project 3: Text-based Game}
    \normalsize
    \\ ~~ \\
    \textbf{Assigned Date: Thursday 03 April} \\
    \textbf{Due Date: Wednesday 16 April}
    
    \section*{Objectives}
    \paragraph{}Create a simple, text-based adventure game.
    
    \subsection*{Starter Code \& Assignment}
    \paragraph{}Using the starter code below (or code attached here \attachfile{../../Code/proj3_game.py}), design a game! The game isn't very interesting, and as-is, it doesn't inform the player of connected rooms or available items. That would be a good feature to implement.
    
    \paragraph{}Be creative. You're free to change the settings, items, layout, commands, inventory, and other aspects.
    
    \begin{lstlisting}[language=python]
# Name: Alice Smith (your name here!)

def show_instructions():
    print("""
Text-Based Adventure Game
=========================
Commands:
    go [direction]
    get [item]
    use [item]""")

def show_status():
    print("---------------------------")
    print("You are in the " + current_room)
    print(rooms[current_room]['description'])
    print("Inventory: " + str(inventory))
    print("---------------------------")

# Game setup
inventory = []
rooms = {
    'Hall': {
        'description': 'a long hallway with paintings on the walls.',
        'items': [],
        'actions': {
            'north': 'Kitchen',
            'east': 'Living Room'
        }
    },
    'Kitchen': {
        'description': 'a kitchen with a fridge and a table.',
        'items': ['key'],
        'actions': {
            'south': 'Hall'
        }
    },
    'Living Room': {
        'description': 'a cozy living room with a fireplace.',
        'items': [],
        'actions': {
            'west': 'Hall',
            'north': 'Garden'
        }
    },
    'Garden': {
        'description': 'a beautiful garden with blooming flowers.',
        'items': [],
        'actions': {
            'south': 'Living Room'
        }
    }
}

current_room = 'Hall'

show_instructions()

# Game loop
while True:
    show_status()

    # Get the player's next move
    move = input("> ").lower().split()

    # Check if the player wants to move
    if move[0] == 'go':
        direction = move[1]
        if direction in rooms[current_room]['actions']:
            current_room = rooms[current_room]['actions'][direction]
        else:
            print("You can't go that way.")

    # Check if the player wants to get an item
    elif move[0] == 'get':
        item = move[1]
        if item in rooms[current_room]['items']:
            inventory.append(item)
            rooms[current_room]['items'].remove(item)
            print("You picked up the " + item + ".")
        else:
           print("There is no " + item + " here.")

    # Implement more commands and game logic here

    # Example ending condition
    if current_room == 'Garden' and 'key' in inventory:
        print("You used the key to unlock the gate and escaped, you win!")
        break
    \end{lstlisting}
    
    \section*{Reminder}
    \paragraph{}Assignments are your own effort. Do not share your code.
    
    \section*{Grading Rubric}
    \begin{itemize}
        \item Game works: 75 points.
        \item Invalid input properly handled: 25 points.
    \end{itemize}
    
    \section*{What to Submit}
    \begin{itemize}
        \item Your Python script.
    \end{itemize}
    
\end{document}
