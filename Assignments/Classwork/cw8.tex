% Copyright 2002-2024 The University of Maryland Baltimore County (UMBC)
% 1000 Hilltop Circle, Baltimore, Maryland, 21250, USA
% https://www.csee.umbc.edu/

\documentclass[letter,10pt]{article}
\usepackage[breaklinks]{hyperref}
\hypersetup{
    bookmarks=true,         % show bookmarks bar?
    unicode=false,          % non-Latin characters in Acrobat’s bookmarks
    pdftoolbar=true,        % show Acrobat’s toolbar?
    pdfmenubar=true,        % show Acrobat’s menu?
    pdffitwindow=false,     % window fit to page when opened
    pdfstartview={XYZ null null 1.00},    % disable zoom
    pdftitle={Classwork 7},    % title
    pdfauthor={Richard Zak},     % author
    pdfsubject={UMBC CMSC104 Problem Solving and Computer Programming},   % subject of the document
    pdfkeywords={Computer Science, Programming, Problem Solving, CSEE}, % list of keywords
    pdfnewwindow=true,      % links in new PDF window
    colorlinks=false,       % false: boxed links; true: colored links
    linkcolor=red,          % color of internal links (change box color with linkbordercolor)
    citecolor=green,        % color of links to bibliography
    filecolor=magenta,      % color of file links
    urlcolor=cyan           % color of external links
}
\usepackage{graphicx}
\usepackage{fancyhdr}
\usepackage{multicol}
\pagestyle{fancy}
\usepackage[letterpaper, margin=1in]{geometry}
\geometry{letterpaper}
\usepackage{listings} % Syntax highlighing
\usepackage{xcolor}
\usepackage{parskip} % Disable initial indent
\usepackage{color,soul} % Highligher
\usepackage[normalem]{ulem} % Strikethrough with \sout{}

\definecolor{codegreen}{rgb}{0,0.6,0}
\definecolor{codegray}{rgb}{0.5,0.5,0.5}
\definecolor{codepurple}{rgb}{0.58,0,0.82}
\definecolor{backcolour}{rgb}{0.97,0.97,0.97}

\lstdefinestyle{mystyle}{
    backgroundcolor=\color{backcolour},
    commentstyle=\color{codegreen},
    keywordstyle=\color{magenta},
    numberstyle=\tiny\color{codegray},
    stringstyle=\color{codepurple},
    basicstyle=\ttfamily\small,
    breakatwhitespace=false,
    breaklines=true,
    captionpos=b,
    keepspaces=true,
    numbers=left,
    numbersep=5pt,
    showspaces=false,
    showstringspaces=false,
    showtabs=false,
    tabsize=2
}

\lstset{style=mystyle}

\usepackage[utf8]{inputenc}
\fancyhf{}
\renewcommand{\headrulewidth}{0pt} % Remove default underline from header package
\rhead{CMSC 104 Section 01: Classwork 8}
%\rhead{}
\lhead{\begin{picture}(0,0) \put(0,-10){\includegraphics[width=1.1cm]{../../Images/UMBC-vertical}} \end{picture}}
\cfoot{\thepage}
\rfoot{\input{../../semester}}
\lfoot{CMSC 104 Section 01}
\AtEndDocument{\vfill \footnotesize{Last modified: 20 August 2024}}
\AtEndDocument{\rfoot{\input{../../semester}}}
\renewcommand\thesubsection{\arabic{subsection}} % Show only subsection numbers, not section.subsection

\begin{document}
    
    \huge
    \textbf{Classwork 8: Class Grades Simulator}
    \normalsize
    \\ ~~ \\
    \textbf{In-class Date: Thursday 17 April} \\
    \textbf{Due Date: Wednesday 23 April}
    
    \section*{Objectives}
    \paragraph{}Practice writing a program which uses classes \& loops.
    
    \section*{The Assignment}
    \paragraph{}Write a program to simulate class letter grades (A, B, C, D, or F) and output the count of each letter grade to the screen. To do this, you will utilize a for loop that calculates a random letter grade each time through the loop and update the appropriate counter variable using the += assignment operator. Make sure to check for invalid values. Outside the loop, you will print the final results.
    
    \subsection*{Example Output}
    \begin{verbatim}
    [rzak1@linux1 cw8]$ python3 grades.py
    Out of 36 students, here is the class grade breakdown:
    A: 10
    B: 8
    C: 10
    D: 3
    F: 5
    [rzak1@linux1 cw8]$ 
    \end{verbatim}
    
    \subsection*{Starter Code}
    \paragraph{}Use this code to help you get started.
    
    \begin{lstlisting}[language=python]
# Name: Alice Smith (your name here!)
        
class Grades:
    def __init__(self):
        # Something has to be initialized
    
    def add_grade(self, grade):
        # Add the grade to the list
    
    def average(self):
        avg = 0.0
        for ?? in ???:
            # add the grade to the running average
    
        return ???
    
    def size(self):
        return ????
    
    def summary(self):
        A = 0
        B = 0
        C = 0
        D = 0
        F = 0
    
        for ?? in ???:
            # Several if statements go here
    
        print("Grades summary:")
        print(f"\tAs: {A}")
        print(f"\tBs: {B}")
        print(f"\tCs: {C}")
        print(f"\tDs: {D}")
        print(f"\tFs: {F}")
    
    
if __name__ == '__main__':
    import random
    
    size = 36
    
    # Don't change anything below
    for i in range(size):
        g = random.randint(0, 100)
        grades.add_grade(g)
    
    print("Class size: {}".format(grades.size()))
    print("Average: %1.2f" % grades.average())
    grades.summary()
    \end{lstlisting}
    
    \section*{Reminder}
    \paragraph{}Assignments are your own effort. Do not share your code.
    
    \subsection*{Extra Credit}
    \paragraph{}Try to do the following embellishments:
    
    \paragraph{Option 1:}Instead of using the same size class each time, use a pseudo-random generator function to decide.
    
    \paragraph{Option 2:}Allow the user to decide between choosing the size of the class or using a random number.
    
    \paragraph{Option 3:}Print the lowest and the highest grades.
    
    \paragraph{Option 4:}Add support for +/- grades.
    
    \section*{Grading Rubric}
    \begin{itemize}
        \item Runs: 50 points
        \item Proper logic: 50 points
        \begin{itemize}
            \item EC1: +5 points
            \item EC2: +5 points
            \item EC3: +5 points
            \item EC4: +5 points
        \end{itemize}
    \end{itemize}
    
    \section*{What to Submit}
    \begin{itemize}
        \item Your Python script.
    \end{itemize}
\end{document}