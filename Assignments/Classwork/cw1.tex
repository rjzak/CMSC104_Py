% Copyright 2020-2024 Richard J. Zak
% richard.j.zak@gmail.com

\documentclass[letter,10pt]{article}
\usepackage[breaklinks]{hyperref}
\hypersetup{
    bookmarks=true,         % show bookmarks bar?
    unicode=false,          % non-Latin characters in Acrobat’s bookmarks
    pdftoolbar=true,        % show Acrobat’s toolbar?
    pdfmenubar=true,        % show Acrobat’s menu?
    pdffitwindow=false,     % window fit to page when opened
    pdfstartview={XYZ null null 1.00},    % disable zoom
    pdftitle={Classwork 1},    % title
    pdfauthor={Richard Zak},     % author
    pdfsubject={UMBC CMSC104 Problem Solving and Computer Programming},   % subject of the document
    pdfkeywords={Computer Science, Programming, Problem Solving, CSEE}, % list of keywords
    pdfnewwindow=true,      % links in new PDF window
    colorlinks=false,       % false: boxed links; true: colored links
    linkcolor=red,          % color of internal links (change box color with linkbordercolor)
    citecolor=green,        % color of links to bibliography
    filecolor=magenta,      % color of file links
    urlcolor=cyan           % color of external links
}
\usepackage{graphicx}
\usepackage{fancyhdr}
\usepackage{multicol}
\pagestyle{fancy}
\usepackage[letterpaper, margin=1in]{geometry}
\geometry{letterpaper}
\usepackage{parskip} % Disable initial indent
\usepackage{color,soul} % Highligher
\usepackage[normalem]{ulem} % Strikethrough with \sout{}

\usepackage[utf8]{inputenc}
\fancyhf{}
\renewcommand{\headrulewidth}{0pt} % Remove default underline from header package
\rhead{CMSC 104 Section \input{../../section}: Classwork 1}
%\rhead{}
\lhead{\begin{picture}(0,0) \put(0,-10){\includegraphics[width=1.1cm]{../../Images/UMBC-vertical}} \end{picture}}
\cfoot{\thepage}
\rfoot{\input{../../semester}}
\lfoot{CMSC 104 Section \input{../../section}}
\AtEndDocument{\vfill \footnotesize{Last modified: 28 July 2024}}
\AtEndDocument{\rfoot{\input{../../semester}}}
\renewcommand\thesubsection{\arabic{subsection}} % Show only subsection numbers, not section.subsection

\begin{document}
    \huge
    \textbf{Classwork 1: Introduction to Python \& IDLE}
    \normalsize
    \\ ~~ \\
    \textbf{In-class Date: Wednesday 03 September} \\
    \textbf{Due Date: Tuesday 09 September}
    
    \section*{Objectives}
    \paragraph{}To become familiar with IDLE, a Python development environment.
    
    \section*{Assignment: First Python Script with IDLE}
    \subsection{Install Python}
    \paragraph{}As mentioned in the syllabus, ensure that Python with IDLE is installed on your computer, available at \url{https://www.python.org/downloads/}.
    
    \subsection{Use IDLE as a text editor}
    \paragraph{}Run IDLE on your computer. It'll be in the Start Menu if you have a Windows machine, or in ``/Applications/Python 3.12/IDLE.app'' if you have a Mac (the version number might be slightly different and that's okay).
    
    \begin{enumerate}
        \item In IDLE, go to ``File'' $\rightarrow$ ``New File''. A new window will appear.
        \item Using a \texttt{for} loop, write code which does the following:
        \begin{itemize}
            \item Print the phrase ``I will not surf the web in class'' ten times.
            \item Print integers from 0 to 9.
            \item Print integers from 0 to 10.
            \item Print only the even integers from 0 to 20.
        \end{itemize}
        \item Test your script by running it, go to ``Run'' $\rightarrow$ ``Run Module'', or press F5.
        \begin{itemize}
            \item Fix any errors.
        \end{itemize}
        \item Add a comment line at the top with your name. Comments start with \texttt{\#}.
        \item Save your script as a .py file.
        \item Submit your assignment on Blackboard.
    \end{enumerate}

    \section*{Reminder}
    \paragraph{}Assignments are your own effort. Do not share your code.

    \section*{Grading Rubric}
    \paragraph{}Script prints:
    \begin{itemize}
        \item the phase above: 25 points.
        \item integers 0..9: 25 points.
        \item integers 0..10: 25 points.
        \item even integers: 25 points.
    \end{itemize}
    
    \section*{What to Submit}
    \begin{itemize}
        \item Your saved Python script.
    \end{itemize}
    
\end{document}