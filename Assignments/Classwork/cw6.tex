% Copyright 2002-2024 The University of Maryland Baltimore County (UMBC)
% 1000 Hilltop Circle, Baltimore, Maryland, 21250, USA
% https://www.csee.umbc.edu/

\documentclass[letter,10pt]{article}
\usepackage[breaklinks]{hyperref}
\hypersetup{
    bookmarks=true,         % show bookmarks bar?
    unicode=false,          % non-Latin characters in Acrobat’s bookmarks
    pdftoolbar=true,        % show Acrobat’s toolbar?
    pdfmenubar=true,        % show Acrobat’s menu?
    pdffitwindow=false,     % window fit to page when opened
    pdfstartview={XYZ null null 1.00},    % disable zoom
    pdftitle={Classwork 6},    % title
    pdfauthor={Richard Zak},     % author
    pdfsubject={UMBC CMSC104 Problem Solving and Computer Programming},   % subject of the document
    pdfkeywords={Computer Science, Programming, Problem Solving, CSEE}, % list of keywords
    pdfnewwindow=true,      % links in new PDF window
    colorlinks=false,       % false: boxed links; true: colored links
    linkcolor=red,          % color of internal links (change box color with linkbordercolor)
    citecolor=green,        % color of links to bibliography
    filecolor=magenta,      % color of file links
    urlcolor=cyan           % color of external links
}
\usepackage{graphicx}
\usepackage{fancyhdr}
\usepackage{multicol}
\pagestyle{fancy}
\usepackage[letterpaper, margin=1in]{geometry}
\geometry{letterpaper}
\usepackage{listings} % Syntax highlighing
\usepackage{xcolor}
\usepackage{parskip} % Disable initial indent
\usepackage{color,soul} % Highligher
\usepackage[normalem]{ulem} % Strikethrough with \sout{}

\definecolor{codegreen}{rgb}{0,0.6,0}
\definecolor{codegray}{rgb}{0.5,0.5,0.5}
\definecolor{codepurple}{rgb}{0.58,0,0.82}
\definecolor{backcolour}{rgb}{0.97,0.97,0.97}

\lstdefinestyle{mystyle}{
    backgroundcolor=\color{backcolour},
    commentstyle=\color{codegreen},
    keywordstyle=\color{magenta},
    numberstyle=\tiny\color{codegray},
    stringstyle=\color{codepurple},
    basicstyle=\ttfamily\small,
    breakatwhitespace=false,
    breaklines=true,
    captionpos=b,
    keepspaces=true,
    numbers=left,
    numbersep=5pt,
    showspaces=false,
    showstringspaces=false,
    showtabs=false,
    tabsize=2
}

\lstset{style=mystyle}

\usepackage[utf8]{inputenc}
\fancyhf{}
\renewcommand{\headrulewidth}{0pt} % Remove default underline from header package
\rhead{CMSC 104 Section 01: Classwork 6}
%\rhead{}
\lhead{\begin{picture}(0,0) \put(0,-10){\includegraphics[width=1.1cm]{../../Images/UMBC-vertical}} \end{picture}}
\cfoot{\thepage}
\rfoot{Spring 2025
}
\lfoot{CMSC 104 Section 01}
\AtEndDocument{\vfill \footnotesize{Last modified: 04 August 2024}}
\AtEndDocument{\rfoot{Spring 2025
}}
\renewcommand\thesubsection{\arabic{subsection}} % Show only subsection numbers, not section.subsection

\begin{document}
    
    \huge
    \textbf{Classwork 6: Blackjack Strategy}
    \normalsize
    \\ ~~ \\
    \textbf{In-class Date: Tuesday 25 March} \\
    \textbf{Due Date: Monday 31 March}
    
    \section*{Objectives}
    \paragraph{}Practice using \texttt{if} statements.
    
    \section*{Assignment Hit or Stand}
    \paragraph{}For this assignment, you will write a program that tells the user to ``hit'' or ``stand'' in a game of Blackjack (also known as Twenty-one).
    \paragraph{}Blackjack is a casino card game where the objective is to have the cards you are dealt total up as close to 21 as possible. If you go over 21 (a bust), you lose. The cards are from a standard deck (most casinos use several decks at once). Cards 2-10 have the values shown. Face cards (Jack, Queen, and King) have value 10. An Ace is either 1 or 11, whichever is to your advantage.
    \paragraph{}Each player is initially dealt two cards face up. The dealer is given 1 card face up and 1 card face down. Then, each player gets one turn to ask for as many extra cards as desired, one at a time. To receive another card, the player ``hits''. When no more cards are wanted, the player ``stands''. Wikipedia has a more comprehensive description of the game \url{https://en.wikipedia.org/wiki/Blackjack}.
    \paragraph{}The strategy that you will implement is a rather simple one. You will probably lose money slowly in a casino if you follow this strategy. (If you don't follow a strategy like this one, you will lose money \textit{quickly}).
    \begin{itemize}
        \item If your cards total 17 or higher, always stand regardless of what the dealer is showing in their face-up card.
        \item If your cards total 11 or lower, always hit.
        \item If your cards add up to 13 to 16 (inclusive), hit if the dealer is showing 7 or higher, otherwise stand.
        \item If your cards add up to 12, hit unless the dealer is showing 4 to 6 (inclusive). In that case, stand.
    \end{itemize}
    
    \paragraph{}The sample runs of your program should look like this:
    \begin{verbatim}
        [rzak1@linux1 cw5]$ python blackjack.py
        Here's some advice for blackjack.
        Tell me which card the dealer is showing.
        Enter 2-9, 10 (Jack, Queen, King) or 11 (Ace): 7
        What is the combined total of your hand? 12
        You should: Hit
        [rzak1@linux1 cw5]$ python blackjack.py
        Here's some advice for blackjack.
        Tell me which card the dealer is showing.
        Enter 2-9, 10 (Jack, Queen, King) or 11 (Ace): 5
        What is the combined total of your hand? 12
        You should: Stand
        [rzal1@linux1 cw5]$ python blackjack.py
        Here's some advice for blackjack.
        Tell me which card the dealer is showing.
        Enter 2-9, 10 (Jack, Queen, King) or 11 (Ace): 11
        What is the combined total of your hand? 19
        You should: Stand
    \end{verbatim}
    
    \section*{Notes}
    \begin{itemize}
        \item You may have to use \texttt{if} statements inside another \texttt{if} statement.
    \end{itemize}
    
    \subsection*{Starter Code \& Assignment}
    \paragraph{}Use this code to help you get started.
    \begin{lstlisting}[language=python]
# Name: Alice Smith (your name here!)

dealerFaceUpCard = 0 # value of dealer's face up card
handTotal = 0 # total value of player's two cards

# Game intro prompt - greet the user

# Prompt for dealer face up card

# While loop until valid value (2 - 11) is given
while dealerFaceUpCard < 2 || dealerFaceUpCard > 11:
    ???

# Prompt for user's hand total
handTotal = ???

# While loop until valid value (4 - 22) is given
while handTotal < 4 || handTotal > 22:
    ???

# If user's cards total 17 or higher, always stand;
# else if user's cards total 11 or lower, always hit;
# else if user's cards add up to 13 to 16 (inclusive),
#   if dealer is showing 7 or higher, hit;
#   else stand;
# else if user's cards add up to 12,
#   if dealer is showing 4 to 6 (inclusive), stand;
#   else hit;
    \end{lstlisting}

    \section*{Reminder}
    \paragraph{}Assignments are your own effort. Do not share your code.
    
    \section*{Grading Rubric}
    \begin{itemize}
        \item Runs: 45 points
        \item Correct logic: 55 points
    \end{itemize}
    
   \section*{What to Submit}
   \begin{itemize}
       \item Your Python script.
   \end{itemize}
\end{document}
