% Copyright 2020-2024 Richard J. Zak
% richard.j.zak@gmail.com

\documentclass[letter,10pt]{article}
\usepackage[breaklinks]{hyperref}
\hypersetup{
    bookmarks=true,         % show bookmarks bar?
    unicode=false,          % non-Latin characters in Acrobat’s bookmarks
    pdftoolbar=true,        % show Acrobat’s toolbar?
    pdfmenubar=true,        % show Acrobat’s menu?
    pdffitwindow=false,     % window fit to page when opened
    pdfstartview={XYZ null null 1.00},    % disable zoom
    pdftitle={Classwork 3},    % title
    pdfauthor={Richard Zak},     % author
    pdfsubject={UMBC CMSC104 Problem Solving and Computer Programming},   % subject of the document
    pdfkeywords={Computer Science, Programming, Problem Solving, CSEE}, % list of keywords
    pdfnewwindow=true,      % links in new PDF window
    colorlinks=false,       % false: boxed links; true: colored links
    linkcolor=red,          % color of internal links (change box color with linkbordercolor)
    citecolor=green,        % color of links to bibliography
    filecolor=magenta,      % color of file links
    urlcolor=cyan           % color of external links
}
\usepackage{graphicx}
\usepackage{fancyhdr}
\usepackage{multicol}
\pagestyle{fancy}
\usepackage[letterpaper, margin=1in]{geometry}
\geometry{letterpaper}
\usepackage{listings} % Syntax highlighing
\usepackage{xcolor}
\usepackage{parskip} % Disable initial indent
\usepackage{color,soul} % Highligher
\usepackage[normalem]{ulem} % Strikethrough with \sout{}

\definecolor{codegreen}{rgb}{0,0.6,0}
\definecolor{codegray}{rgb}{0.5,0.5,0.5}
\definecolor{codepurple}{rgb}{0.58,0,0.82}
\definecolor{backcolour}{rgb}{0.97,0.97,0.97}

\lstdefinestyle{mystyle}{
    backgroundcolor=\color{backcolour},
    commentstyle=\color{codegreen},
    keywordstyle=\color{magenta},
    numberstyle=\tiny\color{codegray},
    stringstyle=\color{codepurple},
    basicstyle=\ttfamily\small,
    breakatwhitespace=false,
    breaklines=true,
    captionpos=b,
    keepspaces=true,
    numbers=left,
    numbersep=5pt,
    showspaces=false,
    showstringspaces=false,
    showtabs=false,
    tabsize=2
}

\lstset{style=mystyle}

\usepackage[utf8]{inputenc}
\fancyhf{}
\renewcommand{\headrulewidth}{0pt} % Remove default underline from header package
\rhead{CMSC 104 Section 02
: Classwork 3}
%\rhead{}
\lhead{\begin{picture}(0,0) \put(0,-10){\includegraphics[width=1.1cm]{../../Images/UMBC-vertical}} \end{picture}}
\cfoot{\thepage}
\rfoot{Spring 2025
}
\lfoot{CMSC 104 Section 02
}
\AtEndDocument{\vfill \footnotesize{Last modified: 28 July 2024}}
\AtEndDocument{\rfoot{Spring 2025
}}
\renewcommand\thesubsection{\arabic{subsection}} % Show only subsection numbers, not section.subsection

\begin{document}
    \huge
    \textbf{Classwork 3: Modeling \& Simulation}
    \normalsize
    \\ ~~ \\
    \textbf{In-class Date: Wednesday 17 September} \\
    \textbf{Due Date: Tuesday 23 September}
    
    \section*{Objectives}
    \paragraph{}To continue getting used to writing Python code.
    
    \section*{Assignment:}
    \subsection{Install Python}
    \paragraph{}As mentioned in the syllabus, ensure that Python with IDLE is installed on your computer, available at \url{https://www.python.org/downloads/}. Optionally, you may use PyCharm Community Edition available at \url{https://www.jetbrains.com/pycharm/}, or Google's interactive Python environment at \url{https://colab.research.google.com/}.
    
    \subsection*{Starter Code \& Assignment}
    \paragraph{}Use the starter code below and follow the in-line instructions.
    
    \begin{lstlisting}[language=python]
# Name: Alice Smith (your name here!)

# Part 1: Doubling the allowance. Starting from a penny per day, calculate
# how much allowance you'd make on any given day, where the user is prompted
# for the day. Day 1: $0.01, Day 2: $0.02, Day 3: $0.04, Day 4: $0.08, etc.

start = 0.01

# prompt the user for a number to represent a day
day = ??

# calculate the total with a loop
for ???:
    ???

# Part 2: Tip Calculator. Write code which prompts the user for a sub-total
# and tip percentage, then display the final bill.
# Extra credit: handle both situations where the tip might be greater or less than
# one. For example, 10% could be 10.0 or 0.10.
# Remember to use the correct data type.

subtotal = ???
tip = ???
total = ?
print("Grand total:", total)

# Part 3: Fibonacci calculator. The Fibonacci sequence starts with 0 and 1, and
# continues by adding the prior two digits. 0 and 1 become 1; 1 and 1 become 2; etc.
# The first few numbers in the sequence are: 1, 2, 3, 5, 8, 13, 21, 34, 55, 89.
# Write code to primpt the user for a number, and print the Fibonacci number at
# that position in the sequence.

num1 = 0
num2 = 1
count = ???
for ???:
    ???
    temp = num1 + num2
    num1 = num2
    num2 = temp

print("The number at position", count, " is", num2)

# Part 4: Prime numbers. As the user for a number and test that it's a prime
# number. Do this by checking if the user's number is divisible by any of the
# numbers between 2 and the user's number minus one. As a shortcut, you may check
# numbers between 2 and the square root if the user's number.

import math # for sqrt()

the_num = ???
is_prime = True

for _ in range(2, math.sqrt(the_num)):
    # Test that the number is prime, hint: use the % operator.
    
    # Update `is_prime` appropriately.
    # Exit early if isn't not prime.

if is_prime:
    print(the_num, "is prime!")
else:
    print(the_num, "is not prime.")
        
    \end{lstlisting}
    
    \section*{Reminder}
    \paragraph{}Assignments are your own effort. Do not share your code.
    
    \section*{Grading Rubric}
    \paragraph{}Script prints:
    \begin{itemize}
        \item allowance for a given day: 25 points.
        \item the correct total with tip: 25 points.
        \begin{itemize}
            \item Tip extra credit: +5 points.
        \end{itemize}
        \item the Fibonacci value at the $n^{th}$ th position in the sequence: 25 points.
        \item when the user's number is prime correctly: 25 points.
    \end{itemize}
    
    \section*{What to Submit}
    \begin{itemize}
        \item Your saved Python script.
    \end{itemize}
    
\end{document}