% Copyright 2020-2024 Richard J. Zak
% richard.j.zak@gmail.com

\documentclass[letter,10pt]{article}
\usepackage[breaklinks]{hyperref}
\hypersetup{
    bookmarks=true,         % show bookmarks bar?
    unicode=false,          % non-Latin characters in Acrobat’s bookmarks
    pdftoolbar=true,        % show Acrobat’s toolbar?
    pdfmenubar=true,        % show Acrobat’s menu?
    pdffitwindow=false,     % window fit to page when opened
    pdfstartview={XYZ null null 1.00},    % disable zoom
    pdftitle={Classwork 10},    % title
    pdfauthor={Richard Zak},     % author
    pdfsubject={UMBC CMSC104 Problem Solving and Computer Programming},   % subject of the document
    pdfkeywords={Computer Science, Programming, Problem Solving, CSEE}, % list of keywords
    pdfnewwindow=true,      % links in new PDF window
    colorlinks=false,       % false: boxed links; true: colored links
    linkcolor=red,          % color of internal links (change box color with linkbordercolor)
    citecolor=green,        % color of links to bibliography
    filecolor=magenta,      % color of file links
    urlcolor=cyan           % color of external links
}
\usepackage{graphicx}
\usepackage{fancyhdr}
\usepackage{multicol}
\pagestyle{fancy}
\usepackage[letterpaper, margin=1in]{geometry}
\geometry{letterpaper}
\usepackage{listings} % Syntax highlighing
\usepackage{xcolor}
\usepackage{parskip} % Disable initial indent
\usepackage{color,soul} % Highligher
\usepackage[normalem]{ulem} % Strikethrough with \sout{}

\definecolor{codegreen}{rgb}{0,0.6,0}
\definecolor{codegray}{rgb}{0.5,0.5,0.5}
\definecolor{codepurple}{rgb}{0.58,0,0.82}
\definecolor{backcolour}{rgb}{0.97,0.97,0.97}

\lstdefinestyle{mystyle}{
    backgroundcolor=\color{backcolour},
    commentstyle=\color{codegreen},
    keywordstyle=\color{magenta},
    numberstyle=\tiny\color{codegray},
    stringstyle=\color{codepurple},
    basicstyle=\ttfamily\small,
    breakatwhitespace=false,
    breaklines=true,
    captionpos=b,
    keepspaces=true,
    numbers=left,
    numbersep=5pt,
    showspaces=false,
    showstringspaces=false,
    showtabs=false,
    tabsize=2
}

\lstset{style=mystyle}

\usepackage[utf8]{inputenc}
\fancyhf{}
\renewcommand{\headrulewidth}{0pt} % Remove default underline from header package
\rhead{CMSC 104 Section 01: Classwork 10}
%\rhead{}
\lhead{\begin{picture}(0,0) \put(0,-10){\includegraphics[width=1.1cm]{../../Images/UMBC-vertical}} \end{picture}}
\cfoot{\thepage}
\rfoot{Spring 2025
}
\lfoot{CMSC 104 Section 01}
\AtEndDocument{\vfill \footnotesize{Last modified: 02 December 2024}}
\AtEndDocument{\rfoot{Spring 2025
}}
\renewcommand\thesubsection{\arabic{subsection}} % Show only subsection numbers, not section.subsection

\begin{document}
    
    \huge
    \textbf{Classwork 10: Bank Accounts}
    \normalsize
    \\ ~~ \\
    \textbf{In-class Date: Wednesday December 4} \\
    \textbf{Due Date: Tuesday December 10}
    
    \section*{Objectives}
    \paragraph{}Practice using Python dictionaries and classes.
    
    \subsection*{Assignment}
    \paragraph{}Using a dictionary, create a simple bank account management application which allows a bank manager to create accounts, and perform certain actions:
    \begin{enumerate}
        \item Create a new account with an initial balance.
        \item View account details (account number, account holder's name, and balance).
        \item Deposit money into an account.
        \item Withdraw money from an account (ensure sufficient balance).
        \item Transfer money between two accounts.
    \end{enumerate}
    
    The bank functionality should be a class, which has methods for performing the needed operations.
    
    \subsection*{Example Output}
    \paragraph{}Use the example output below as a guide to help you develop your application. You don't have to use the exact wording or output, but it should be close.
    
    \begin{verbatim}
Menu:

1. Create Account
2. View Account
3. Deposit
4. Withdraw
5. Transfer
6. Exit

Enter your choice: 1
Enter account holder's name: Alice
Enter initial deposit: 500
Account created successfully! The account number is: 1001

Enter your choice: 1
Enter account holder's name: Bob
Enter initial deposit: 1000
Account created successfully! The account number is: 1002

Enter your choice: 2
Enter the account number: 1002
Account 1002 is owned by Bob, and has a balance of $1000

Enter your choice: 3
Enter account number: 1001
Enter amount to deposit: 300
Deposit successful! New balance: $800

Enter your choice: 4
Enter account number: 1001
Enter amount to withdraw: 100
Withdrawal successful! New balance: $700

Enter your choice: 5
Enter source account number: 1001
Enter destination account number: 1002
Enter transfer amount: 200
Transfer successful! New balances:
  - Source account: $500
  - Destination account: $1200

Enter your choice: 6
Bye!
    \end{verbatim}
    
    \section*{Reminder}
    \paragraph{}Assignments are your own effort. Do not share your code.
    
    \section*{Extra Credit}
    \paragraph{}Write a unit test class in a separate .py file using the Python unit test framework. Read more about it at \url{https://docs.python.org/3/library/unittest.html}. See Classwork 9 for an example.
    
    \section*{Grading Rubric}
    \paragraph{}Script prints:
    \begin{itemize}
        \item Script works as intended: 65 points
        \item Proper data structures (dictionaries): 25 points
        \item Bank functionality encoded as a class: 10 points
        \item Extra Credit total: +40 points
        \begin{itemize}
            \item Unit test class with several tests: +20 points
            \item Unit test class has tests for menu items 1-5: +20 points
        \end{itemize}
    \end{itemize}
    
    \section*{What to Submit}
    \begin{itemize}
        \item Your saved Python script(s).
    \end{itemize}
    
\end{document}
