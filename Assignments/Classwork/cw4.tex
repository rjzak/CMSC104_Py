% Copyright 2020-2024 Richard J. Zak
% richard.j.zak@gmail.com

\documentclass[letter,10pt]{article}
\usepackage[breaklinks]{hyperref}
\hypersetup{
    bookmarks=true,         % show bookmarks bar?
    unicode=false,          % non-Latin characters in Acrobat’s bookmarks
    pdftoolbar=true,        % show Acrobat’s toolbar?
    pdfmenubar=true,        % show Acrobat’s menu?
    pdffitwindow=false,     % window fit to page when opened
    pdfstartview={XYZ null null 1.00},    % disable zoom
    pdftitle={Classwork 4},    % title
    pdfauthor={Richard Zak},     % author
    pdfsubject={UMBC CMSC104 Problem Solving and Computer Programming},   % subject of the document
    pdfkeywords={Computer Science, Programming, Problem Solving, CSEE}, % list of keywords
    pdfnewwindow=true,      % links in new PDF window
    colorlinks=false,       % false: boxed links; true: colored links
    linkcolor=red,          % color of internal links (change box color with linkbordercolor)
    citecolor=green,        % color of links to bibliography
    filecolor=magenta,      % color of file links
    urlcolor=cyan           % color of external links
}
\usepackage{graphicx}
\usepackage{fancyhdr}
\usepackage{multicol}
\pagestyle{fancy}
\usepackage[letterpaper, margin=1in]{geometry}
\geometry{letterpaper}
\usepackage{listings} % Syntax highlighing
\usepackage{xcolor}
\usepackage{parskip} % Disable initial indent
\usepackage{color,soul} % Highligher
\usepackage[normalem]{ulem} % Strikethrough with \sout{}

\definecolor{codegreen}{rgb}{0,0.6,0}
\definecolor{codegray}{rgb}{0.5,0.5,0.5}
\definecolor{codepurple}{rgb}{0.58,0,0.82}
\definecolor{backcolour}{rgb}{0.97,0.97,0.97}

\lstdefinestyle{mystyle}{
    backgroundcolor=\color{backcolour},
    commentstyle=\color{codegreen},
    keywordstyle=\color{magenta},
    numberstyle=\tiny\color{codegray},
    stringstyle=\color{codepurple},
    basicstyle=\ttfamily\small,
    breakatwhitespace=false,
    breaklines=true,
    captionpos=b,
    keepspaces=true,
    numbers=left,
    numbersep=5pt,
    showspaces=false,
    showstringspaces=false,
    showtabs=false,
    tabsize=2
}

\lstset{style=mystyle}

\usepackage[utf8]{inputenc}
\fancyhf{}
\renewcommand{\headrulewidth}{0pt} % Remove default underline from header package
\rhead{CMSC 104 Section 02
: Classwork 4}
%\rhead{}
\lhead{\begin{picture}(0,0) \put(0,-10){\includegraphics[width=1.1cm]{../../Images/UMBC-vertical}} \end{picture}}
\cfoot{\thepage}
\rfoot{Spring 2025
}
\lfoot{CMSC 104 Section 02
}
\AtEndDocument{\vfill \footnotesize{Last modified: 29 July 2024}}
\AtEndDocument{\rfoot{Spring 2025
}}
\renewcommand\thesubsection{\arabic{subsection}} % Show only subsection numbers, not section.subsection

\begin{document}
    
    \huge
    \textbf{Classwork 4: Input \& Output}
    \normalsize
    \\ ~~ \\
    \textbf{In-class Date: Wednesday 01 October} \\
    \textbf{Due Date: Tuesday 07 October}
    
    \section*{Objectives}
    \paragraph{}More practice using \texttt{input()}, \texttt{print()}, variables, and data types.
    
    \subsection*{Starter Code \& Assignment}
    \paragraph{}Using the starter code below, write a program that asks for the user's name, height in inches, and weight in pounds. Your program then replies with the user's name, height in centimeters, and weight in kilograms.
    
    \begin{lstlisting}[language=python]
# Name: Alice Smith (your name here!)
        
# Ask for the person's name.
name = ???
        
# Ask for the height in inches
# Make sure you have the correct data type!
height_in = ???
        
# Ask for the weight in pounds
# Make sure you have the correct data type!
weight_lbs = ???
        
height_cm = 2.54 * ???
weight_km = weight_lbs / 2.205
        
print("Hello", name, "your height is", height_cm, "centimeters, and you weigh", weight_km, "kilograms.")
    \end{lstlisting}
    
    \section*{Reminder}
    \paragraph{}Assignments are your own effort. Do not share your code.
    
    \section*{Extra Credit}
    \paragraph{}Use the appropriate code in \texttt{print()} to only show two values after the decimal when printing the user's height and/or weight.
    
    \section*{Grading Rubric}
    \paragraph{}Script prints:
    \begin{itemize}
        \item accurate centimeters calculation: 50 points.
        \begin{itemize}
            \item Extra Credit: +5 points
        \end{itemize}
        \item accurate kilograms calculation: 50 points.
        \begin{itemize}
            \item Extra Credit: +5 points
        \end{itemize}
    \end{itemize}
    
    \section*{What to Submit}
    \begin{itemize}
        \item Your saved Python script.
    \end{itemize}
    
\end{document}