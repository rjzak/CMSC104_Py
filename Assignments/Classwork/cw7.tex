% Copyright 2002-2024 The University of Maryland Baltimore County (UMBC)
% 1000 Hilltop Circle, Baltimore, Maryland, 21250, USA
% https://www.csee.umbc.edu/

\documentclass[letter,10pt]{article}
\usepackage[breaklinks]{hyperref}
\hypersetup{
    bookmarks=true,         % show bookmarks bar?
    unicode=false,          % non-Latin characters in Acrobat’s bookmarks
    pdftoolbar=true,        % show Acrobat’s toolbar?
    pdfmenubar=true,        % show Acrobat’s menu?
    pdffitwindow=false,     % window fit to page when opened
    pdfstartview={XYZ null null 1.00},    % disable zoom
    pdftitle={Classwork 7},    % title
    pdfauthor={Richard Zak},     % author
    pdfsubject={UMBC CMSC104 Problem Solving and Computer Programming},   % subject of the document
    pdfkeywords={Computer Science, Programming, Problem Solving, CSEE}, % list of keywords
    pdfnewwindow=true,      % links in new PDF window
    colorlinks=false,       % false: boxed links; true: colored links
    linkcolor=red,          % color of internal links (change box color with linkbordercolor)
    citecolor=green,        % color of links to bibliography
    filecolor=magenta,      % color of file links
    urlcolor=cyan           % color of external links
}
\usepackage{graphicx}
\usepackage{fancyhdr}
\usepackage{multicol}
\pagestyle{fancy}
\usepackage[letterpaper, margin=1in]{geometry}
\geometry{letterpaper}
\usepackage{listings} % Syntax highlighing
\usepackage{xcolor}
\usepackage{parskip} % Disable initial indent
\usepackage{color,soul} % Highligher
\usepackage[normalem]{ulem} % Strikethrough with \sout{}

\definecolor{codegreen}{rgb}{0,0.6,0}
\definecolor{codegray}{rgb}{0.5,0.5,0.5}
\definecolor{codepurple}{rgb}{0.58,0,0.82}
\definecolor{backcolour}{rgb}{0.97,0.97,0.97}

\lstdefinestyle{mystyle}{
    backgroundcolor=\color{backcolour},
    commentstyle=\color{codegreen},
    keywordstyle=\color{magenta},
    numberstyle=\tiny\color{codegray},
    stringstyle=\color{codepurple},
    basicstyle=\ttfamily\small,
    breakatwhitespace=false,
    breaklines=true,
    captionpos=b,
    keepspaces=true,
    numbers=left,
    numbersep=5pt,
    showspaces=false,
    showstringspaces=false,
    showtabs=false,
    tabsize=2
}

\lstset{style=mystyle}

\usepackage[utf8]{inputenc}
\fancyhf{}
\renewcommand{\headrulewidth}{0pt} % Remove default underline from header package
\rhead{CMSC 104 Section 01: Classwork 7}
%\rhead{}
\lhead{\begin{picture}(0,0) \put(0,-10){\includegraphics[width=1.1cm]{../../Images/UMBC-vertical}} \end{picture}}
\cfoot{\thepage}
\rfoot{Spring 2025
}
\lfoot{CMSC 104 Section 01}
\AtEndDocument{\vfill \footnotesize{Last modified: 20 August 2024}}
\AtEndDocument{\rfoot{Spring 2025
}}
\renewcommand\thesubsection{\arabic{subsection}} % Show only subsection numbers, not section.subsection

\begin{document}
    
    \huge
    \textbf{Classwork 7: I Am Thinking of a Number}.
    \normalsize
    \\ ~~ \\
    \textbf{In-class Date: Tuesday 01 April} \\
    \textbf{Due Date: Monday 07 April}
    
    \section*{Objectives}
    \paragraph{}Practice writing a program that uses if statements and a while loop.
    
    \section*{The Assignment}
    \paragraph{}Write a program to play the game ``I'm thinking of a number.'' The program will play the role of the person who has the ``secret'' number. Your program should prompt the user to guess a number. If user's guess is incorrect, your program should say whether the guess is too high or too low, and try again.
    
    \subsection*{Example Output}
    \begin{verbatim}
        [rzak1@linux1 cw7]$ python3 thinking.py
        I'm thinking of a number between 1 and 100.
        Guess my number.
        Your guess? 13
        Too low!
        Your guess? 20
        Too low!
        Your guess? 35
        Too low!
        Your guess? 99
        Too high!
        Your guess? 74
        Too high!
        Your guess? 45
        Too low!
        Your guess? 84
        Too high!
        Your guess? 60
        Too low!
        Your guess? 70
        Too high!
        Your guess? 65
        Too high!
        Your guess? 63
        Too low!
        Your guess? 64
        You got it!
        [rzak1@linux1 cw7]$ 
    \end{verbatim}
    
    \subsection*{Starter Code}
    \paragraph{}Use this code to help you get started. You must use a loop in your program which terminates when the user guesses the correct number. Warn the user if they enter a number less than 1 or a number greater than 100. Notice how the starting point file uses a constant \texttt{SECRET\_NUMBER} that holds the number your program is ``thinking'' of. That way, if you want/need to change the secret number, you only have to change it in one place.
    \begin{lstlisting}[language=python]
# Name: Alice Smith (your name here!)

# Constant value to use if you don't want to do any Extra Credit
SECRET_NUMBER = 24

# Variable to use without any Extra Credit embellishments
guess = 0 #user's guess

# Variable(s) to use for the Extra Credit embellishments
min = 0 # lowest number of range to guess
max = 0 # highest number of range to guess
secretNumber = 0 # number to guess

# EC1: Seed the random number generator

# EC2: Prompt for minimum value of range to guess

# EC2: Prompt for maximum value of range to guess

# EC2: Ensuring max > min

# EC1 & EC2: Generate secret number within range

# Prompt user to guess number
# NOTE: Need to use min and max instead of 1 and 100 for EC
guess = ???

# While loop until user guesses the secret number
while True:
    # NOTE: Need to use secretNumber instead of SECRET_NUMBER for EC
    # If user's guess is not in range, display "Guess must be between low and high!",
    # replacing low and high accordingly
    # else if user's guess is greater than secret number, print "Too high!"
    # else if user's guess is less than secret number, print "Too low!"
    # Prompt user for new guess
    # "break" when done.

# The user guessed the secret number, display "You got it!\n"
    \end{lstlisting}

    \section*{Reminder}
    \paragraph{}Assignments are your own effort. Do not share your code.
    
    \subsection*{Extra Credit}
    \paragraph{}Try to do the following embellishments:
    
    \paragraph{Option 1:}Instead of picking the same secret number each time, use a pseudo-random generator function. To do so, you must include the following module:
    \begin{lstlisting}[language=python]
        import random
    \end{lstlisting}
    
    \paragraph{}Now each call to the function \texttt{random.randint()} will return a pseudo-random number\footnote{\url{https://en.wikipedia.org/wiki/Pseudorandom\_number\_generator}} as an integer value. It requires two arguments, the lower and higher values. To get a number between 1 and 100: \texttt{n = random.randint(1, 100)}. Then no matter what random returns, the variable n is assigned a number between 1 and 100.
    
    \paragraph{Option 2:}Ask the user for the range of numbers from which to choose the secret number (instead of always choosing a number between 1 and 100). This will affect how you implement Option 1.
    
    \section*{Grading Rubric}
    \begin{itemize}
        \item Runs: 50 points
        \item Proper logic: 50 points
        \begin{itemize}
            \item EC1: +5 points
            \item EC2: +5 points
        \end{itemize}
    \end{itemize}
    
    \section*{What to Submit}
    \begin{itemize}
        \item Your Python script.
    \end{itemize}
\end{document}