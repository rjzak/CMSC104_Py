% Copyright 2024 Richard J. Zak
% richard.j.zak@gmail.com

\documentclass[letter,10pt]{article}
\usepackage[breaklinks]{hyperref}
\hypersetup{
    bookmarks=true,         % show bookmarks bar?
    unicode=false,          % non-Latin characters in Acrobat’s bookmarks
    pdftoolbar=true,        % show Acrobat’s toolbar?
    pdfmenubar=true,        % show Acrobat’s menu?
    pdffitwindow=false,     % window fit to page when opened
    pdfstartview={XYZ null null 1.00},    % disable zoom
    pdftitle={Final Exam Review},    % title
    pdfauthor={Richard Zak},     % author
    pdfsubject={UMBC CMSC104 Problem Solving and Computer Programming},   % subject of the document
    pdfkeywords={Computer Science, Programming, Problem Solving, CSEE}, % list of keywords
    pdfnewwindow=true,      % links in new PDF window
    colorlinks=true,       % false: boxed links; true: colored links
    linkcolor=blue,          % color of internal links (change box color with linkbordercolor)
    citecolor=green,        % color of links to bibliography
    filecolor=magenta,      % color of file links
    urlcolor=cyan           % color of external links
}
\usepackage{graphicx}
\usepackage{fancyhdr}
\usepackage{multicol}
\pagestyle{fancy}
\usepackage[letterpaper, margin=1in]{geometry}
\geometry{letterpaper}
\usepackage{parskip} % Disable initial indent
\usepackage{color,soul} % Highligher
\usepackage[normalem]{ulem} % Strikethrough with \sout{}
\usepackage{placeins} % FloatBarrier
\usepackage[utf8]{inputenc}
\usepackage{listings} % Syntax highlighing

% arrays table
\usepackage{multirow,tabularx}
\newcolumntype{Y}{>{\centering\arraybackslash}X}
\renewcommand{\arraystretch}{2}

% header in tables
\newcommand*{\thead}[1]{\multicolumn{1}{c}{\bfseries #1}}

\fancyhf{}
\renewcommand{\headrulewidth}{0pt} % Remove default underline from header package
\rhead{CMSC 104 Section 01: Final Exam Review}
%\rhead{}
\lhead{\begin{picture}(0,0) \put(0,-10){\includegraphics[width=1.1cm]{../Images/UMBC-vertical}} \end{picture}}
\cfoot{\thepage}
\rfoot{Spring 2025
}
\lfoot{CMSC 104 Section 01}
\AtEndDocument{\vfill \footnotesize{Last modified: 03 May 2025}}
\AtEndDocument{\rfoot{Spring 2025
}}
\renewcommand\thesubsection{\arabic{subsection}} % Show only subsection numbers, not section.subsection
\title{Final Exam Review}

\begin{document}
%\maketitle
\huge
\textbf{Final Exam Review}
\normalsize

\tableofcontents

\section{Computer Components}
\paragraph{}A computer is a device which performs general-purpose computation on numbers. A computer is ``general purpose'' because the same machine may be used for different purposes: video games, communication (email, messaging), home banking, business purposes (spreadsheets, databases, etc), industrial/manufacturing, etc. A small computer could be used at home to browse the internet, or to control a pump for a water system.

\paragraph{}Regarding numbers, it is through the meaning we give those numbers that they become data, and ultimately become useful. A computer takes many forms: cell phones, laptops, desktops, servers, control systems in automobiles \& aircraft, video game consoles, smart watches, and many more. They vary drastically in price, from a \$35 Raspberry Pi\footnote{Raspberry Pi website: \url{https://www.raspberrypi.org/}}, which is about the size of a credit card and geared for students \& hobbyists, to multi-million dollar government-operated supercomputers\footnote{\url{https://en.wikipedia.org/wiki/Supercomputer}}.

\subsection{Storage}
\paragraph{}Storage on a computer is thought of as long or short-term storage.
\begin{itemize}
    \item Short-term (or ephemeral) storage: \hyperref[sec:ram]{RAM} is gone when the computer turns off.
    \item Long-term (or permanent) storage: hard drives, removable media, and other devices which hold data even when the computer is powered off, or when removed from the computer.
\end{itemize}

\paragraph{}Long-term storage can be a hard drive, USB thumb drive, CD or DVD optical disc, floppy disks, or tape back-up systems. \textit{Short-term storage (memory) is ALWAYS faster}, and the long-term storage devices are ranked from fastest to slowest:
\begin{enumerate}
    \item Hard drive (solid state disk or ``SSD'')
    \item Hard drive (spinning magnetic disk or ``HDD'')
    \item USB thumb/flash drive
    \item CD or DVD optical drive
    \item Tape - slow, but that's fine for backing up data since it's seldom used (hopefully)
\end{enumerate}

\paragraph{}Computer memory and these long-term storage devices provide \texttt{random access}, meaning it is possible to read any single byte from any location without having to read \texttt{all} of the preceding bytes. The exception: tape back-up. To retrieve some data at the end of the tape, the entirety of the tape's contents must be read! This, and because the tape systems are slow, makes reading from tape data very slow.

\subsection{Input/Output Devices}
\paragraph{}Simply put, input devices receive data, and output devices send data to the user. Examples:
\begin{itemize}
    \item Input: keyboard, mouse, scanner, webcam
    \item Output: monitor, graphics card, printer
    \item Input \& Output: network card, audio card (audio out to speakers, microphone input), USB devices
\end{itemize}

\subsection{Central Processing Unit (CPU)}
\paragraph{}Made up of billions of transistors yet taking up less space than a fingernail, the modern CPU can be thought of as the ``brain'' of the computer. It provides instruction to the rest of the devices, and ensures that the devices communicate with each other.

\paragraph{}Processors have memory of their own, called registers and cache. These are the fastest forms of memory on a computer, though management of this memory is solely managed by the processor itself and the size of this memory is rather small compared to the RAM available on most systems.

\subsection{Random Access Memory}\label{sec:ram}
\paragraph{}System memory is in the form of memory sticks connected to the motherboard, sometimes permanently connected. These circuits are fast, temporary storage, and are used by the operating system to keep the computer running, and by running applications to perform the tasks asked of them. More RAM in a computer generally makes the computer \texttt{seem} faster, as more things can be done without having to temporarily save data to disk. This is done so that the computer may function even when the system memory is exhausted, but since hard drives are much slower than RAM, it seems to take longer to use applications when this happens.

\paragraph{}What makes it \textbf{random}? Data in memory may be accessed directly. Like opening a book to a specific page versus having to open an ancient scroll and continuously search forward until the desired data is found.

\section{Operating Systems}
\paragraph{}The operating system is a program which manages the computer's functionality on behalf of the user, and is the user's interface to the hardware. The operating system manages the devices to ensure sound comes out of the speakers from a music player, for example, as well as runs programs on behalf of the user. If the CPU is the ``brain'' of the hardware, the operating system is the ``brain'' of the software.

\section{Semantics vs Syntax}
\paragraph{Semantics} The meaning or purpose of a phrase, sentence, of segment of code is referred to as semantics. Example: \textit{The dog caught the ball} vs. \textit{The ball walked the dog.}. The second example doesn't make sense, despite following the common form of Subject Verb Object. A dog would catch a ball, but a ball doesn't do anything to a dog.

\paragraph{Syntax} The manner in which code or human language is structured is syntax. Example: \textit{The dog caught the ball.} vs. \textit{dog ball caught the the}. Here the words are in a random order (whereas the prior example was in the correct order but was still non-sense).

\paragraph{}Semantics vs. syntax are based on the language. A pattern used by one language won't likely be valid for another language, and the same is true of programming languages.

\paragraph{Note:}The Python interpreter (and compilers) will catch syntax errors. Semantic errors, also sometimes called bugs\footnote{Computer bugs got their name from a time in September 1947, where a computer problem was caused by an actual insect. \url{https://www.atlasobscura.com/places/grace-hoppers-bug}.}, are not caught automatically. You have to identify them, and figure out how to resolve them. If someone could invent a semantic error detector which works with a compiler, that person would be a billionaire. Since the computer has no knowledge of what you the programmer is trying to do, or the problem being solved, only the human can notice the semantic error.

\section{Algorithms}
\paragraph{}An algorithm is a set of instructions which is clear \& easy to understand no matter who reads it, and has a small (or manageable) amount of steps. It is clear at each step what must be done, without assumptions or confusion.

\paragraph{}A professor at Harvard University made a video which demonstrates this clearly by discussing how to make a peanut butter and jelly sandwich: \url{https://youtu.be/okkIyWhN0iQ}. Note that some steps, like ``open the bag'', or how to open the bag of bread, is missed because some people \textit{assumed} aspects of that step based on prior knowledge. A good algorithm wouldn't have an assumption, each step would be clear. If an assumption can't be avoided, it would be documented somewhere in the code. An algorithm should pay attention to deals, and handle unexpected inputs accordingly.

\paragraph{}A clear and concise definition:
\begin{quote}
An algorithm is a finite set of unambiguous, executable instructions which directs a terminating activity.
\end{quote}

\section{Pseudocode}
\paragraph{}To help with planning how a program, or function might look like, \texttt{pseudocode} is used because it hides the issues pertaining to syntax. Separating out semantics from syntax can be helpful in planning how the code will look by first focusing on what it's supposed to do. It can also be used to explain an algorithm to someone who isn't familiar with code. Example:

\begin{verbatim}
    DISPLAY "How many grades?"
    READ <numberOfGrades>
    counter = 0
    sum = 0
    WHILE (<counter> < <numberOfGrades>)
        READ <grade>
        <sum> = <sum> + <grade>
    END_WHILE
    average = <sum> / <numberOfGrades>
    DISPLAY "Class average", <average>
\end{verbatim}

\section{Programming in Python}
\paragraph{}Python is an interpreted language, which means that you have to have Python installed to run (or execute) a Python application (or script). This is in contrast to compiled languages, such as C/C++, Fortran, Go, Rust, which contain code which are compiled into a stand-alone application. The person running a program which is compiled does not need the compiler installed to run the program.

\subsection{Variables \& Data Types}
\paragraph{}Data is stored in a variable, much like in algebra. The variable represents some value, such as $x = 5$ means that referencing $x$ will result in the variable being treated as if it were the number five, until the value is altered. In addition to the value, a variable also has a data type. Some basic data types are:

\begin{itemize}
    \item Integer: a numeric value which does not have value after the decimal point. Example: $3$.
    \item Float: a floating point value is a number which does have value after the decimal point. Example: $3.141592653$.
    \item Character: a value representing one character, such as ``A''.
    \item String: an array or sequence of many characters, such as ``Apples are delicious''.
    \item Boolean: a value which represents only True or False.
\end{itemize}

\FloatBarrier
\subsection{Mathematical Operations}
\paragraph{}Math operations in Python (and other programming languages) are processed in a similar manner was algebra. There is an order of operations, from first to last:

\begin{enumerate}
    \item Anything in parenthesis, inner most parentheses to outer.
    \item Exponents
    \item Multiplication, division, modulo in order from left to right.
    \item Addition \& subtraction, in order from left to right
    \item Assignment, with the expression on the right being stored in the variable on the left.
\end{enumerate}

\begin{figure}[htbp]
\centering
\begin{tabular}{ c c c }
 Operator & Description & Example \\
 \hline 
 **    & Exponentiation & \verb|2**4 == 16| \\
 $*$ & Multiplication & $3 * 5 = 15$ \\  
 /     & Division & $15 / 2 = 7.50$ \\
 //    & Integer division & $15 / 2 = 7$ \\
 \%  & Modulo (remainder) & $15 \% 2 = 1$ \\
 +    & Addition & $5 + 10 = 15$ \\
 -    & Subtraction & $5 - 10 = -5$ \\
 += & Increment (add to a variable) & $a += 3$ \\
 -=  & Decrement (subtract from a variable) & $a -= 3$
\end{tabular}
\caption{Python's Numeric Operators}
\end{figure}

\FloatBarrier
\paragraph{}Python's \verb|math| library has several math operations, including but not limited to:
\begin{itemize}
    \item \verb|math.sqrt()|: square root
    \item \verb|math.sin()|: sine function using radians
    \item \verb|math.cos()|: cosine function using radians
    \item \verb|math.log()|: base 10 logarithm
\end{itemize}

\subsection{Boolean Logic}
\paragraph{}Boolean expressions evaluate to \verb|True| or \verb|False|. A simple expression might be \verb|x > 0|, which is true if $x = 9$, but false if $x = -30$, for example. Using keywords \verb|and|, \verb|or|, and \verb|not|, two or more expressions may be combined into larger, more complex expressions. These expressions may be used with loops and conditional statements.

\paragraph{}Boolean algebra examples:
\begin{itemize}
    \item Anything \verb|or|'d with true is true. \verb|a or true == true|.
    \item Both \verb|and| and \verb|or| distribute. \verb|a or (b and c) == (a or b) and (a or c)|
    \item Double negatives cancel out: \verb|not(not a) == a|
    \item DeMorgan's law: \\
        \verb|not(a or b) == (not a) and (not b)| \\
        \verb|not(a and b) == (not a) or (not b)|
\end{itemize}

\subsection{Loops}
\paragraph{}Code is executed by three basic patterns:
\begin{enumerate}
    \item Sequence: code is run in the order it's presented in the source code (or script), top to bottom.
    \item Repetition: code can be repeatedly run without having to type it out several times, typically in a loop but can also be a recursive function.
    \item Selection or Branching: decisions can be made so that some part of the code may or may not be run, or there may be sections where it's one section \textit{or} another section, based on some criteria.
\end{enumerate}

\paragraph{}Loops are a way to have the same code run for several \textit{iterations}. There are several types of loops depending on certain criteria.
\begin{itemize}
    \item Definitive loop: A loop, typically using the \verb|for| keyword, where it is known a head of time how many iterations the loop will have. Maybe it's to average the scores of 10 grades, or to iterate over all the elements in a list.
    \item Indefinite loop: A loop, typically using the \verb|while| keyword, where it is not known how many iterations the loop will have. Maybe ten, maybe a thousand, maybe zero. An example might be input validation, where you ask the user to correct invalid input, but don't need to seek correction if the user entered correct data the first time.
    \begin{itemize}
        \item Sentinel loop: A special case for indefinite loops, where there will be some unknown amount of data being collected, but a special case of the data is a signal to end the loop, and is not a piece of data. An example might be calculating a class average, where all positive values are accepted as grades, but a negative value instead means the loop is finished.
    \end{itemize}
\end{itemize}

\paragraph{}Sometimes, it's necessary to exit a loop early, which uses the \verb|break| statement, or to skip an iteration with the \verb|continue| statement. Break \& continue are often used when there's an error, or sometimes Break is used when no further computation is needed to solve the task at hand.

\subsubsection{Recursion}
\paragraph{}A function which calls itself is a type of loop, though without the normal loop-related keywords. The basic structure of recursive functions is to have a value which decreased (or shrinks, if it's a list), and to have a hard-coded stopping condition (where the recursive function returns a value instead of calling itself again).

\subsection{Conditionals}
\paragraph{}Sometimes, depending on some condition(s), code should or should not run. This decision takes the form of an \verb|if| statement. There may be a code segment which runs \verb|if| a condition is true, and there might be other code which runs after the \verb|else| keyword if the condition in the \verb|if| was false. The keyword \verb|elif| is a shortcut for \verb|else if|, allowing the construction of a conditional block with several conditions (not just ``this'' or ``that'').

\subsection{Lists, Sets, Tuples, \& Dictionaries}
\paragraph{}Collections of data accessible under one variable name allows for easier access to several items. Lists, Sets, and Tuples allow for variables to be stored and accessed by \textit{indexing} into it. So a list \verb|x = [0] * 100| contains one hundred zeros, which is manageable, unlike having to store a zero in one hundred separate variables. The difference between them is that Sets only allow unique items, tuples are immutable (read-only), and lists have no restrictions.

\paragraph{}Dictionaries can be thought of as lists which allow indexing by a string instead of an integer. They are ideal for non-sequential data, and can be faster for accessing an element.

\paragraph{}For all of these types (lists, sets, tuples, dictionaries), the contained data can be anything: more lists, sets, tuples, classes, etc. A ``list of lists'' might be a way to implement a matrix, for example.

\subsection{Functions}
\paragraph{}Functions are reusable segments of code which serve a few purposes:
\begin{itemize}
    \item Code reuse: functions can be implemented for code which gets uses a lot, reducing the lines of code and making the project easier to manage. If some part of this frequent code had to be changed for some reason, if it's in a function, the change theoretically only has to be made in one place.
    \item Code sharing: \verb|print| is a function used quite frequently, and yet we don't have to know how it works, and we didn't have to write it. Code can be shared so that others may use classes \& functions created by others without having to learn exactly how it works to use. It's like using a laptop without needing the knowledge to build one yourself. This results in more time to work on the problem you're trying to solve instead of re-solve problems already figured out.
    \item Readability: code can get rather complicated. In some cases, it might be beneficial to pull out some code segment, move it to its own function (which has a name), so that the code is a little more understandable. This doesn't decrease the lines of code, but \textit{may} improve understandability.
\end{itemize}

\paragraph{}Finally, functions themselves are treated as variables in Python, such that a function may be a parameter to a function. This was discussed in the context of sorting, allowing for the sorting of a list of tuples or list of classes. Which element (in the tuple or class) is to be used as the basis for sorting the list? A function as a parameter can solve this, allowing the use of Python's built-in \verb|sort| function instead of having to recreate it.

\subsection{Classes}
\paragraph{}Classes can be thought of a software object to model something, such that this ``something'' has data (or attributes or characteristics) and functions/methods (things it \textit{does}). There are a few class-related concepts to remember:

\begin{itemize}
    \item Encapsulation: Encapsulation links related data and functions/methods which are strongly related. This makes it easier
to manage these related things by accessing them by a single variable, the class instance. It also serves to provide a usable object type to the developer and hide implementation details, something the developer using the class doesn't need to know about (and may be confusing).
    \item Polymorphism: Polymorphism enables a portion of code to call similar functions (the same interface) which actually
calls very different class types. See Chapter 12’s description of the Poker application, which allows
the one program to update a graphical application or a command line application. In this instance, we
have one set of code which gets to do two things - reducing the amount of code required to have this
advanced functionality.
    \item Interface: The interface is an agreed-upon set of functions/methods and data types used between two or more classes for interaction. This allows for interaction between different, possibly unknown, types and helps with the building of robust software. \href{https://en.wikipedia.org/wiki/API}{Wikipedia: API}.
    \item Inheritance: Inheritance allows for a class to be created which inherits functions/methods and variables of another
class, the parent or super class. This allows for code reduce by objects, so a child class may use the
data and functions of a parent class without having to reimplement or copy that code into the child
class.
\end{itemize}

\paragraph{}Classes have functions, sometimes called ``methods'', which represent what a class can do, and how the data is manipulated though the course of the program running. Some functions are special:

\begin{itemize}
    \item Constructor: \verb|def __init__(self...)| creates the class, and may take parameters needed to build the class' initial state.
    \item Accessor (getter): functions which access some internal data of the class, but don't allow changes.
    \item Mutator (setter): functions which allow changes to the internal data of the class, but often have \verb|if| statements to ensure the data is valid, and that the rest of the class' internal data is also updated accordingly, if needed.
    \item Methods: functions which allow the class to ``do'' something related to the purpose of the class.
    \item Private: functions which begin with an underscore are considered private to the function and should not be accessed by the developer using the function. These functions may change the class' internal state in an inconsistent way if used by functions other than the class' public functions.
\end{itemize}

\subsection{Modules}
\paragraph{}Python has several built-in \textit{modules}, such as \verb|math|, \verb|sys|, and many others. Python can also access elements from other Python files through this module system. For example, if you had a class called \verb|Grades| in a file called \verb|school.py|, a Python file in the same directory could access that class by first importing the class \verb|from school import Grades|, or by with \verb|import school|, and later using \verb|school.Grades| to call the class' constructor.

\paragraph{}Python has a rich ecosystem of third-party modules created by people around the world, and they are usually listed on \url{https://pypi.org}. Python has a tool called \verb|pip| which makes it easy to install modules from PyPI.

\subsection{Testing}
\paragraph{}Some say code doesn't work until it's tested and found to be correct. Testing in this case refers to unit tests, which have some sample input for a function, some expected output from a function, and ensure the function under test returns the expected output. A proper test shows an error message if the expected and received data returned aren't a match.

\paragraph{}Unit tests are especially helpful for a few reasons:
\begin{itemize}
    \item Unit tests ensure that the function being tested works correctly. This assumes that the function is being tested with a few input types, including invalid inputs to test that the function has an error when it should.
    \item Having unit tests can help ensure that a program still works properly when a big change as been made (that nothing has broken in the process of development).
    \item When working on a team, unit tests can ensure that new code contributions don't break the project.
    \item Often, projects use third-party modules, and unit tests can ensure that updates to these modules don't break the project.
\end{itemize}

\section{Software Design}
\paragraph{}Designing software is a complicated task, and there are many ways to do this. The decision usually comes down to preference, though the nature of the project or the problem being solved may dictate a style over another.

\begin{itemize}
    \item Top-Down: Work on the program's starting point first, breaking the problem into smaller pieces and gradually implementing \& connecting the pieces until the project works.
    \item Bottom-Up: Work on the small items first, gradually working up to the main program.
    \item Prototyping: Work on parts of the program, keeping the program functioning though incomplete. Fill in the gaps as development progresses. Start with a much easier goal and fewer features, test that it works, add more features.
\end{itemize}

\end{document}